\documentclass[11pt]{article}
%\usepackage{uvatoc} % replace this line with the one below for your submission
\usepackage[response]{uvatoc}



\begin{document}

\makeheader

\makemytitle{Week 7: Loony Automata Tune}
\submitter {Benjamin (aqn9yv, dlb2ru, ht6xd, iad4de, jmn4fms, lw7jz)}

\directions{
\collaboration{You should work on the problems yourself, before discussing with
others, and with your cohorts are your cohort meeting. By the Assessed Cohort Meeting,
you and all of your cohortmates, should be prepared to present and discuss solutions to
all of the assigned problems (including the programming problems). In addition to discussing with your cohortmates, you may
discuss the problems with anyone you want, and use any resources you want except for
any materials from previous offerings of this course, which are not permitted.}
}



\begin{problem}
NFA Size
\end{problem}
\directions{
Regular expressions are represented by strings of text. Consider that we have a regular expression that is $n$ characters long. 

Show that the language represented by this regular expression can be computed by a NFA with $O(n)$ states (that is, the function from the length of the a regular expression to the number of states needed for an NFA that decides the language represented by that regular expression is in $O(n)$ where $n$ is the length of the input).
}


\newtheorem{theorem}{Theorem}

\begin{theorem}
The language represented by a regular expression that is $n$ characters long can be computed by a non-deterministic finite automaton of $O(n)$ states.
\end{theorem}


\begin{proof}
A regular expression consisting only of the empty string can be represented by a NFA of 1 state. The language represented by a regular expression consisting of one literal character can be computed by a NFA of 2 states.

Note that the burden of proof of the following results was achieved in lecture.

Let $R_{1}$ and $R_{2}$ be regular expressions such that $R_{1}$ is $n_{1}$ characters long and computed by an NFA with $n_{1} + c_{1}$ states and $R_{2}$ is $n_{2}$ characters long and computed by an NFA with $n_{2} + c_{2}$ states.

The NFA which computes $(R_{1}|R_{2})$ requires one additional state with $\varepsilon$ transitions to each of the start states for $R_{1}$ and $R_{2}$. Thus $(R_{1}|R_{2})$ requires $n_{1} + n_{2} + c_{1} + c_{2} + 1$ states, where $(R_{1}|R_{2})$ is $n_{1} + n_{2} + 1$ characters long. This can be simplified by stating the Regex $(R_{1}|R_{2})$ which is n bits long is computed in $n + c$ states where $n = n_{1} + n_{2}$ and c is some constant.

The concatenation of $R_{1}$ and $R_{2}$, represented by $R_{1}R_{2}$, is formed by simply adding an epsilon transition from the final states of $R_{1}$ to the start states of $R_{2}$, and thus does not require any additional states besides those used to compute $R_{1}$ and $R_{2}$. Thus for the FSA which computes the expression $R_{1}R_{2}$ of length n requires n states, where $n = n_{1} + n_{2}$.

The Kleene star * adds one character to the regular expression, $R_{1}^*$, and adds a new start state to the NFA for the empty string that has an $\varepsilon$ transition to the original start state. Thus $R_{1}^*$ will be $n_{1} + 1$ characters long, and take $n_{1} + c + 1$ states to compute, where $R_{1}$ took $n_{1} + c$ states.

Any regular expression is a combination of the empty string, literal characters, unions or Kleene stars, and as shown above any regular expression of length n composed of these elements will be computed by an FSA with $n + c$ states, where c is some constant resulting from the operations of alternation and Kleene star.

Let $R$ be a regular expression with $n$ characters, and computed by an FSA with $n+c$ states.
To show that the number of states of a regular expression, $R$, is in $O(n)$ where n is the number of characters in R, we need to show that for some constant $C$, that the number of states is at most $C*n$.

From our results above we have that the number of states is $n + c$ for some constant $c$, and thus it suffices to show that $n + c \leq C*n, \hspace{1mm} \forall n \geq n_{0}$ for some $n_{0} \in \N$.

Let $n_{0} = c/(C-1)$, then for $n \geq n_{0}$, $n \geq c/(C-1)$
\\$\implies$ $(C-1) \cdot n \geq c$
\\$\implies$ $C\cdot n - n \geq c$
\\$\implies$ $C \cdot n \geq n + c$ $\iff$ $n + c \leq C \codt n$
\\$\implies$ $n+c \in O(n)$
\\$\implies$ The number of states required to compute a regular expression is in $O(n)$.
\end{proof}





\end{document}