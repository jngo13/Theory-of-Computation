\documentclass[11pt]{article}
%\usepackage{uvatoc} % replace this line with the one below for your submission
\usepackage[response]{uvatoc}



\begin{document}

\makeheader

\makemytitle{Week 5: Complexity Sells Better}
\submitter {Benjamin (aqn9yv, dlb2ru, ht6xd, iad4de, jmn4fms, lw7jz)}
\directions{
\collaboration{You should work on the problems yourself, before discussing with
others, and with your cohorts are your cohort meeting. By the Assessed Cohort Meeting,
you and all of your cohortmates, should be prepared to present and discuss solutions to
all of the assigned problems (including the programming problems). In addition to discussing with your cohortmates, you may
discuss the problems with anyone you want, and use any resources you want except for
any materials from previous offerings of this course, which are not permitted.}
}





\begin{problem}
Little-$o$
\end{problem}
\directions{Another useful notation is ``little-$o$'' which is designed to capture the notion that a function $g$ grows much faster than $f$:

\begin{definition}[$o$]
A function $f(n): \mathbb{N} \rightarrow \mathbb{R}$ is in $o(g(n))$ for any function $g(n): \mathbb{N} \rightarrow \mathbb{R}$ if and only if for every positive constant $c$, there exists an $n_0 \in \mathbb{N}$ such that:
$$ \forall n > n_0 . f(n) < cg(n).$$
\end{definition}

In other words, $f(n) \in o(g(n))$ provided that no matter what positive constant that is chosen for $c$, eventually there comes a point were $c\cdot g(n) > f(n)$ forever more.

Provide a proof for each of the following sub-problems.}

\begin{enumerate}[(a)]

\item Prove that for any function $f$, $f \notin o(f)$.

\begin{proof}

$f \in o(f) \iff \forall c > 0,  \exists n_{0}$ such that $\forall n \in \N$ where $n \textgreater n_{0}, f(n) < c \cdot f(n)$ 

Thus, it is sufficient to provide a counterexample where, for some $c > 0$, $f(n) \geq c \cdot f(n)$.

Choose $c = 1$.

In this case, $\forall n \in \N$, $f(n) < 1 \cdot f(n)$ is never true, since $f(n) = f(n)$.

$\implies \exists c > 0$ such that $f(n) < c \cdot f(n)$ is false

$\implies \exists c > 0$ such that $f(n) \geq c \cdot f(n)$

$\implies f(n) \notin o(f(n))$

\end{proof}

\item Prove that $n \in o(n \log n)$.

\begin{proof}
Assume $log_{b} n$ has arbitrary base b. Let $c > 0$ an arbitrary constant.

\begin{align*}
&n \textless c \cdot n \cdot \log(n) \\
\iff& 1 \textless c \cdot \log(n)\\
\iff& \frac{1}{c} \textless \log(n)\\
\iff& b^{\frac{1}{c}} \textless n\\
\iff& n \textgreater b^{\frac{1}{c}}\\
\end{align*}
Thus, choose $n_{0} \in \N$ such that $n_{0} > b^{\frac{1}{c}}$ for arbitrary b, c.
\begin{align*}
&\implies \forall n \in \N, n \geq n_{0}, n > b^{\frac{1}{c}}\\
&\implies \forall n \in \N, n \geq n_{0}, n < c \cdot n \log n, \text{\; as seen above}\\
&\implies \exists n_{0}, \text {\; such that \;} \forall c > 0, n < c \cdot n \log n\\
&\implies n \in o(n \log n)\\
\end{align*}

\end{proof}

\end{enumerate}







\end{document}
