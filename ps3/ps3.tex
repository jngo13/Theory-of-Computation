\documentclass[11pt]{article}
%\usepackage{uvatoc}
% For your submission, you will comment the line above, and uncomment the one below:
\usepackage[response]{uvatoc}

\setlength\parindent{12pt}

\begin{document}

\makeheader

\makemytitle{Week 3: Sweet}

\submitter{Benjamin (aqn9yv, dlb2ru, ht6xd, iad4de, jmn4fms, lw7jz)}

\collaborators{Definition of Boolean circuits from \emph{Introduction to Theoretical Computer Science} by Boaz Barak}

\directions{
This is a template to help with your write-up for Week 3. The actual problem you will write up will be selected by your Cohort Leader at the Assessed Cohort Meeting.
}

\directions{

\subsection{Clone the Problem Set 3 Template Repository}

See the \href{https://uvatoc.github.io/ps/ps1template.pdf}{Week 1 template for directions} on Getting Started with LaTeX. Similarly to Week 1, one member of your cohort should create a copy of the Problem Set 3 repository, by following these steps (we recommend doing this together, with the one creating the repository sharing her screen for everyone to follow along):

\begin{enumerate}
\item Download the Problem Set 3 template from: \url{https://uvatoc.github.io/ps/ps3.zip}
\item In Overleaf, click on \keyword{Create First Project} or \keyword{New Project} in Overleaf and select \keyword{Upload Project} from the menu.
\item Click \keyword{Select a .zip file} and then select the \keyword{ps3.zip} file you downloaded in step 1.
\item Share the repository with your cohortmates by clicking the "Share" button at the top right of the overleaf window, and entering your cohortmates email addresses in the sharing form.
\end{enumerate}

Click on \keyword{ps3.tex} to see the LaTeX source for this file, which is the file you will modify to prepare your solution. The first thing you should do in \keyword{ps3.tex} is set up your cohort name as the author of the submission by replacing the line, \texttt{\textbackslash submitter\{TODO: your name\}}, with your the name of your cohort (e.g., \texttt{\textbackslash submitter\{Cohort Hopper (Ada Lovelace, Don Knuth)\}}). For the list of cohort members, this should usually be everyone in your cohort, but if someone did not contribute during the week, they should not be included in your submission list (and should have informed us about their absence separately).

Before submitting your \keyword{ps3.pdf} file, also remember to:
\begin{itemize}
\item List your collaborators and resources, replacing the TODO in {\texttt{\textbackslash collaborators\{TODO: replace ...\}}} with your collaborators and resources. You do not need to include

\item Replace the second line in \keyword{ps3.tex}, \texttt{\textbackslash usepackage\{uvatoc\}} with \texttt{\textbackslash usepackage[response]\{uvatoc\}} so the directions do not appear in your final PDF. You can do this by using the LaTeX comment token, {\texttt{\%}}. The rest of the line after a {\texttt{\%}} is treated as a comment. You'll notice after you to this, when you Recompile the document, most of it will disappear (everything in \keyword{\textbackslash directions} is left out, so only your solution will appear in the submitted document).
\end{itemize}
}

\stepcounter{problem}  % skip Problem 1

\begin{problem}
Maximum number of Inputs (Induction Practice)
\end{problem}
\directions{
The \emph{depth} of a circuit is the length of the longest path (in the number of gates) from the an input to an output in the circuit. Prove using induction that the maximum number of inputs for a Boolean circuit (as defined by Definition 3.5 in the book) that produces one output that depends on all of its inputs with depth $d$ is $2^{d}$ for all $d \ge 0$. (Note: there are ways to prove this without using induction, but the purpose of this problem is to provide induction practice, so only solutions that are well constructed proofs using the induction principle will be worth full credit.)
}


\begin{definition}

Let $n$, $m$, $s$ be positive integers with $s \geq m$. A Boolean circuit with $n$ inputs, $m$ outputs, and $s$ gates, is a labeled directed acyclic graph (DAG) $G=(V,E)$ with $s+n$ vertices. There are $n$ inputs of the circuit, and each gate $s$ can perform either a unary or binary operation, i.e. it can have either one or two inputs.

\end{definition}


\newtheorem{theorem}{Theorem}


\begin{theorem}
The maximum number of inputs for a Boolean circuit that produces one output that depends on all of its inputs with depth $d$ is $2^{d}$ for all $d \ge 0$.
\end{theorem}


\begin{proof} We will proceed by induction on the depth $d$ of the Boolean circuit.

% i think base case starts at 0 let me fix this
\paragraph{Base Case} Consider a circuit of depth $d = 0$. This implies that the circuit has 1 output, and no gates. Since the output should depend on all the inputs, this implies that the circuit has only 1 input. In this case, $1 \le 2^d = 2^0 = 1$.


\paragraph{Inductive Hypothesis} We now assume that the number of inputs in a circuit with depth d is bounded above by $2^{d}$, in the hopes of concluding that a circuit with depth d+1 takes at most $2^{d+1}$ inputs. 
\par We begin by noting at a depth of d+1, the longest path from an input to the final output consists of d+1 gates. The $(d+1)^{th}$ gate must take at most 2 inputs, one of which is a circuit of depth d. To achieve a maximal number of inputs, assume the $(d+1)^{th}$ gate takes two inputs from the outputs of two circuits of depth d, which we will call "subcircuits." 
\par We know from our assumption that the number of inputs leading into both subcircuits are at most $2^{d}$. Therefore the circuit with depth d+1 will include at most $2^{d} + 2^{d}$ inputs. Therefore, if we denote the number of inputs by $i_{d+1}$ then we have
\begin{center} 
 $i_{d+1} \leq 2^{d} + 2^{d} = 2 \cdot 2^{d} = 2^{d+1}$.
\end{center}

\paragraph{Conclusion}
Thus we have shown in our base case that for a depth of $d=0$, $i_{d} \leq 2^{d}$, and from the induction step, given $i_{d} \leq 2^{d}$, we have that $i_{d+1} \leq 2^{d+1}$. So by the principle of induction we have proved that a circuit with a depth of d includes at most $2^{d}$ inputs, \forall $d$ \in\N.
\\
\end{proof}

\end{document}
